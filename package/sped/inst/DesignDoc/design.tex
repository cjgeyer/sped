
\documentclass[11pt]{article}

\usepackage{amsmath}
\usepackage{amsthm}
\usepackage{indentfirst}
\usepackage{natbib}
\usepackage{url}

\newcommand{\REVISED}{\begin{center} \LARGE REVISED DOWN TO HERE \end{center}}
\newcommand{\MOVED}[1][equation]{\begin{center} [#1 moved] \end{center}}

\newtheorem{theorem}{Theorem}

\begin{document}

\title{An R Package for Multigene Descent Probabilities}

\author{Charles J. Geyer}

\maketitle

\section{Introduction}

Here we design an R package that incorporates some of the functionality
of a very old (and defunct) S package for statistical genetics described
by \citet{geyer}.  In particular, our new package will do multigene
descent probabilities as described originally by \citet{thompson} and
as implemented by \citet{geyer}.

\section{Pedigrees}

We work relative to a defined pedigree in which every individual has either
two parents or none specified.  Those with none specified are called
\emph{founders}.  A pedigree may be specified by a \emph{triplets matrix}
having three columns and each row gives the names of a non-founder individual,
its father, and its mother, in that order.  We check that no individual is
its own ancestor.  Optionally, we check that sexes are consistent
(no individual is both father and mother).  This check is optional
so that we can deal with hermaphroditic organisms.

Any ancestors of individuals not in the pedigree --- including parents
of founders --- are assumed to not be individuals in the pedigree.
That is, we are assuming that all unknown individuals are not any known
individuals.

\section{Descent Probabilities}

\citet{thompson} defines \emph{multigene descent probabilities}
$g_S(B_1, \ldots, B_n)$ to be the probability that genes
at one autosomal locus randomly chosen from
each of the individuals $B_1$, $\ldots,$ $B_n$
are all descended from genes (not necessarily the same gene)
in some set $S$ of genes in individuals in the pedigree.
The individuals $B_1$, $\ldots,$ $B_n$ need not be distinct.
The set $S$ can be specified by giving for each individual in the pedigree
an integer 0, 1, or 2 that says how many of its genes (at the autosomal locus
in question) are in the set $S$.

Since the order of $B_1$, $\ldots,$ $B_n$ does not matter to the definition,
we may assume these arguments are in sorted order in some order that always
has offspring before parents.  There always is such an order because no
individual can be its own ancestor, and such an order can be found by the
topological sort algorithm \citep[Section~6.6]{aho-et-al}
which is implemented by R function \texttt{tsort} in R package \texttt{pooh}
\citep{pooh-package}.
Thus in what follows we have $B_1 = \cdots = B_r$ for some $r \ge 1$, and
we have $B_1$ not equal to nor an ancestor of $B_i$ for $i > r$.
We also adopt the convention
\begin{subequations}
\begin{equation} \label{eq:g-empty}
   g_S() = 1,
\end{equation}
which makes sense because the
empty set of genes chosen from the empty set of individuals is always contained
in genes descended from $S$ (because the empty set is contained in any set).
\begin{theorem} \label{th:g}
Assume arguments are in an order that has offspring before parents.
If $B_1$ is not a founder, contains no genes of $S$, and occurs only once
in the arguments ($n = 1$ or $B_1 \neq B_2$), then
\begin{equation} \label{eq:g-unique}
   g_S(B_1, \ldots, B_n)
   =
   \tfrac{1}{2} g_S(M_1, B_2, \ldots, B_n)
   + \tfrac{1}{2} g_S(F_1, B_2, \ldots, B_n)
\end{equation}
where $F_1$ is the father of $B_1$ and $M_1$ is the mother of $B_1$.

If $B_1$ is not a founder, contains no genes of $S$, and occurs $r$ times
in the arguments ($B_1 = \cdots = B_r$ and $n = r$ or $B_r \neq B_{r + 1}$),
then
\begin{multline} \label{eq:g-repeat}
   g_S(B_1, \ldots, B_n)
   =
   (\tfrac{1}{2})^{r - 1} g_S(B_1, B_{r + 1}, \ldots, B_n)
   \\
   + \left[ 1 - (\tfrac{1}{2})^{r - 1} \right]
   g_S(M_1, F_1, B_{r + 1}, \ldots, B_n)
\end{multline}
where $F_1$ and $M_1$ are as before.

If $B_1$ is a founder and contains no genes of $S$, then
\begin{equation} \label{eq:g-boundary-zero}
   g_S(B_1, \ldots, B_n) = 0.
\end{equation}

If $B_1$ contains two genes of $S$ and occurs $r$ times in the arguments
($B_1 = \cdots = B_r$ and $n = r$ or $B_r \neq B_{r + 1}$), then
\begin{equation} \label{eq:g-boundary-two}
   g_S(B_1, \ldots, B_n) = g_S(B_{r + 1}, \ldots, B_n).
\end{equation}

If $B_1$ is not a founder, contains one gene of $S$,
and occurs $r$ times in the arguments
($B_1 = \cdots = B_r$ and $n = r$ or $B_r \neq B_{r + 1}$), then
\begin{multline} \label{eq:g-boundary-one-nonfounder}
   g_S(B_1, \ldots, B_n) =
   (\tfrac{1}{2})^r g_S(B_{r + 1}, \ldots, B_n)
   \\
   +
   \tfrac{1}{2} \left[ 1 - (\tfrac{1}{2})^r \right]
   \left[ g_S(F_1, B_{r + 1}, \ldots, B_n)
   + g_S(M_1, B_{r + 1}, \ldots, B_n) \right].
\end{multline}

If $B_1$ is a founder, contains one gene of $S$,
and occurs $r$ times in the arguments
($B_1 = \cdots = B_r$ and $n = r$ or $B_r \neq B_{r + 1}$), then
\begin{equation} \label{eq:g-boundary-one-founder}
   g_S(B_1, \ldots, B_n) =
   (\tfrac{1}{2})^r g_S(B_{r + 1}, \ldots, B_n).
\end{equation}
\end{theorem}
\end{subequations}
\begin{proof}
If $B_1$ is not a founder and $S$ contains no genes of $B_1$
and $B_1 \neq B_2$, then a gene chosen at random from $B_1$ is equally
to have come from $F_1$ or $M_1$ and is equally likely to be either of
the genes in $F_1$ or $M_1$ by Mendel's laws.  Since $B_1$ has no genes
of $S$, the only way it can have genes descended from $S$ is if they
come through $F_1$ or $M_1$.

If $B_1$ is not a founder and $S$ contains no genes of $B_1$
and $B_1 = \cdots = B_r$ and $B_1 \neq B_{r + 1}$ or $r = n$, then
with probability $(\tfrac{1}{2})^{r - 1}$ the same gene is chosen from
$B_1$, $\ldots,$ $B_r$, in which case the probability that this gene
and randomly chosen genes from $B_{r + 1}$, $\ldots,$ $B_n$ are descended
from $S$ is $g_S(B_1, B_{r + 1}, \ldots, B_n)$.  Otherwise, two different
genes are chosen from $B_1$, $\ldots,$ $B_r$, in which case one must be
a randomly chosen gene from $F_1$ and the other must be
a randomly chosen gene from $M_1$ and the probability that these genes
and randomly chosen genes from $B_{r + 1}$, $\ldots,$ $B_n$ are descended
from $S$ is $g_S(M_1, F_1, B_{r + 1}, \ldots, B_n)$.

If $B_1$ is a founder and $S$ contains no genes of $B_1$, then
from the assumption that none of the ancestors of $B_1$ --- all of whom
are unknown ---
are any of the known individuals in the pedigree who collectively contain
the genes in $S$ it follows that $B_1$ cannot contain any genes descended
from $S$.

If $S$ contains two genes of $B_1$
and $B_1 = \cdots = B_r$ and $B_1 \neq B_{r + 1}$ or $r = n$, then
all genes chosen from $B_1$, $\ldots,$ $B_r$ must come from $S$ because
they are chosen from these two genes of $B_1$ contained in $S$.
Hence \eqref{eq:g-boundary-two} holds.

If $B_1$ is not a founder and $S$ contains one gene of $B_1$
and $B_1 = \cdots = B_r$ and $B_1 \neq B_{r + 1}$ or $r = n$, then
with probability $(\tfrac{1}{2})^r$ the genes randomly chosen from
$B_1$, $\ldots,$ $B_r$ are all the one gene of $B_1$ contained in $S$,
in which case the probability that the genes of $B_{r + 1}$, $\ldots,$ $B_n$
are descended from $S$ is $g_S(B_{r + 1}, \ldots, B_n)$.
Otherwise, some of the genes chosen from $B_1$, $\ldots,$ $B_r$ are the
gene of $B_1$ not contained in $S$, in which case this gene is equally
likely to have come from $F_1$ or $M_1$,
in which case the probability that this gene and
the genes of $B_{r + 1}$, $\ldots,$ $B_n$
are descended from $S$ is $g_S(F_1, B_{r + 1}, \ldots, B_n)$
or $g_S(M_1, B_{r + 1}, \ldots, B_n)$.

If $B_1$ is a founder and $S$ contains one gene of $B_1$
and $B_1 = \cdots = B_r$ and $B_1 \neq B_{r + 1}$ or $r = n$, then
with probability $(\tfrac{1}{2})^r$ the genes randomly chosen from
$B_1$, $\ldots,$ $B_r$ are all the one gene of $B_1$ contained in $S$,
in which case the probability that the genes of $B_{r + 1}$, $\ldots,$ $B_n$
are descended from $S$ is $g_S(B_{r + 1}, \ldots, B_n)$.
Otherwise, some of the genes chosen from $B_1$, $\ldots,$ $B_r$ are the
gene of $B_1$ not contained in $S$, in which case this gene came from some
ancestor of $B_1$ --- and all these ancestors are unknown, hence none
are any of the known individuals in the pedigree who collectively contain
the genes in $S$ --- it follows that the gene of $B_1$ not contained in $S$
is not descended from $S$.
\end{proof}

Our equations \eqref{eq:g-unique} through \eqref{eq:g-boundary-one-founder}
are unnumbered displayed equations on pp.~{33--34} in \citet{geyer},
except that two typographical errors have been corrected, one minor (a missing
parenthesis) and one major: what is $1 - (\tfrac{1}{2})^{r - 1}$
in our \eqref{eq:g-repeat} is $2^{r - 1} - 1$ in \citet{geyer} and in
\citet[equation (7)]{thompson}.  This error, if reproduced in the code
for the old S package \texttt{sped}, could have produced probabilities greater
than one in case $B_1 = B_2 = B_3$.  It looks like this error was \emph{not}
reproduced in the code
(see \url{http://users.stat.umn.edu/~geyer/sped/src/gnx.c})
so that code did not have this particular error.
Our equation \eqref{eq:g-empty} is also in \citet{geyer} run into the text
on p.~{34}.

Our \eqref{eq:g-boundary-zero} through \eqref{eq:g-boundary-one-founder}
do not appear in \citet{thompson} who says only
that our \eqref{eq:g-unique} and \eqref{eq:g-repeat} are
are to be used with boundary conditions that specify when individuals
$B_i$ have genes in $S$.
Presumably,
our \eqref{eq:g-empty} and
\eqref{eq:g-boundary-zero} through \eqref{eq:g-boundary-one-founder}
are the boundary
conditions \citeauthor{thompson} referred to, because \citeauthor{geyer}
was \citeauthor{thompson}'s research assistant when \citet{geyer} was written.

\section{Special Descent Probabilities}

These come originally from \citet{thompson-1986}.  We follow \citet{geyer}.

\subsection{Gammas}

The fraction of genes in individual $B$ that comes from founder $A$ is
\begin{equation} \label{eq:gamma}
   \gamma(A, B) = g_{S_A}(B)
\end{equation}
where $S_A$ is the set of genes that contains the two genes of $A$ and no
other genes.

\subsection{Betas}

If, as before, $F_i$ is the father of $B_i$ and $M_i$ is the mother of $B_i$,
then
\begin{equation} \label{eq:beta}
   \beta(A, B_i) = g_{S_A}(F_i, M_i)
\end{equation}
is the bilineal contribution of founder $A$ to individual $B_i$,
the probability that both genes of $B_i$ are descended from genes
of founder $A$.

\subsection{Alphas}

Now let $T_A$ be the set of genes that contains one gene of founder $A$ and no
other genes, and let $F_i$ and $M_i$ be as above, then
\begin{equation} \label{eq:alpha}
   \alpha(A, B_i) = 2 g_{T_A}(F_i, M_i)
\end{equation}
is the inbreeding of individual $B_i$ relative to founder $A$,
the probability that both genes of individual $B$ come from the same
gene in founder $A$.

\subsection{Inbreeding Coefficients}

Then
\begin{equation} \label{eq:inbreeding}
   \alpha(B) = \sum_{A \in \text{Founders}} \alpha(A, B)
\end{equation}

\subsection{Kinship Coefficients}

The kinship coefficient of individuals $B_i$ and $B_j$ is
\begin{equation} \label{eq:kinship}
   \phi(B_i, B_j) = 2 \sum_{A \in \text{Founders}} G_{T_A}(B_i, B_j)
\end{equation}

\begin{thebibliography}{}

\bibitem[Aho, et al.(1983)Aho, Hopcroft, and Ullman]{aho-et-al}
Aho, A.~V., Hopcroft, J.~E., and Ullman, J.~D. (1983).
\newblock \emph{Data Structures and Algorithms}.
\newblock Addison-Wesley, Reading, MA.

\bibitem[Geyer(1988)]{geyer}
Geyer, C.~J. (1988).
\newblock Software for Calculating Gene Survival and Multigene Descent
    Probabilities and for Pedigree Manipulation and Drawing.
\newblock Technical Report No.~153, Department of Statistics,
    University of Washington.
\newblock \url{https://www.stat.washington.edu/article/tech-report/software-calculating-gene-survival-and-multigene-descent-probabilities-and}

\bibitem[Geyer(2017)]{pooh-package}
Geyer, C.~J. (2017).
\newblock R package \texttt{pooh} (Partial Orders and Relations), version 0.3-2.
\newblock \url{http://cran.r-project.org/package=pooh}.

\bibitem[Thompson(1983)]{thompson}
Thompson, E.~A. (1983).
\newblock Gene extinction and allelic origins in complex genealogies
    (with discussion).
\newblock \emph{Proceedings of the Royal Society of London. Series B,
    Biological Sciences}, \textbf{219}, 241--251.
\newblock \url{https://doi.org/10.1098/rspb.1983.0072}.

\bibitem[Thompson(1986)]{thompson-1986}
Thompson, E.~A. (1986).
\newblock Ancestry of alleles and extinction of genes in populations with
    defined pedigrees.
\newblock \emph{Zoo Biology}, \textbf{5}, 161--170.
\newblock \url{https://doi.org/10.1002/zoo.1430050210}.

\end{thebibliography}

\end{document}

